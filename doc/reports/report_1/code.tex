\chapter{Qualité du code (Thibault)}

Ayant déjà codé en C depuis plusieurs années, je savais à quel point il était
nécessaire d'être rigoureux dans l'implémentation de l'OCR.

\epigraph{\textit{« Don't leave “broken windows” (bad designs, wrong decisions,
or poor code) unrepaired. Fix each one as soon as it is discovered.  [...]
neglect accelerates the rot faster than any other factor.»}}{--- \textup{The
Pragmatic Programmer}}

Il fallait tout d'abord se décider d'une \textit{coding style} à respecter,
celle des ACU 2019 a donc été choisie.

L'utilisation régulière de \mintinline{text}{valgrind} a permis de détecter et
régler tous les problèmes liés à la mémoire. Une commande
\mintinline{text}{check-valgrind} a été intégrée au Makefile permettant de
vérifier tout de suite les potentielles fuites de mémoire :

\begin{scriptsize}
\begin{minted}{text}
==7845== HEAP SUMMARY:
==7845==     in use at exit: 49,212 bytes in 238 blocks
==7845==   total heap usage: 2,160,403 allocs, 2,160,165 frees, 34,466,285 bytes allocated
==7845==
==7845== LEAK SUMMARY:
==7845==    definitely lost: 0 bytes in 0 blocks
==7845==    indirectly lost: 0 bytes in 0 blocks
==7845==      possibly lost: 0 bytes in 0 blocks
==7845==    still reachable: 0 bytes in 0 blocks
==7845==                       of which reachable via heuristic:
==7845==                         newarray           : 1,536 bytes in 16 blocks
==7845==         suppressed: 49,212 bytes in 238 blocks
==7845==
==7845== For counts of detected and suppressed errors, rerun with: -v
==7845== ERROR SUMMARY: 0 errors from 0 contexts (suppressed: 18 from 18)
\end{minted}
\end{scriptsize}

\newpage

Une partie du développement du projet fut consacré à la ré-écriture et la
re-structuration du code source. Cette partie est trop souvent négligée et peut
apparaître comme une perte de temps alors que c'est l'exact opposé. Des choix
initiaux de conception seront très souvents erronés, car il est difficile
d'anticiper correctement tous les éléments d'une implémentation. Il faut
corriger ces failles au plus vite pour ne pas dépendre d'elles dans un futur
proche. Garder un code simple, concis, modulaire, et clair a été notre objectif
durant toute cette première période, car ceci est primordial pour le bon
déroulement du projet.
