\chapter{Expériences personnelles}

\begin{itemize}
    \item \textbf{Thibault Allançon} :

        Le langage C était mon premier langage de programmation, je l'apprécie
        donc tout particulièrement d'autant plus que nous l'utilisons pour un
        projet assez conséquent. À la fin de cette première soutenance, notre
        OCR a bien avancé, et le code derrière est vraiment de bonne qualité ce
        qui est très encourageant pour la prochaine soutenance. Il y a une
        grosse différence entre comprendre un réseau de neurones et en
        implémenter un de toutes pièces, cette expérience fût donc très
        formatrice.

    \item \textbf{Quentin Le Helloco} :

        Au début de ce projet, je n’avais aucune connaissance en C, ni en
        réseau de neurones. En revanche, je souhaitais m’orienter dans le
        secteur de l'intelligence artificielle après mes études, c'est pour cela
        que j'ai voulu m'occuper du réseau de notre projet. Les notions abordées
        dans ce thème sont parfois très abstraites et très mathématiques, ce qui
        les rend difficiles à assimiler. Avec Thibault comme chef de groupe, qui
        connaît déjà bien le C et le réseau de neurones, j’ai pu apprendre
        beaucoup. Je sais maintenant davantage à quoi ressemble cet aspect de
        l’informatique qui était assez flou auparavant. J’ai grâce à cela
        accepté un stage dans le machine learning pour en apprendre plus dans ce
        domaine. Ce projet me permettra d’avoir des bases lors de mon stage.

    \newpage

    \item \textbf{Timothé Helme-Guizon} :

        Lors de ce projet, j'ai souvent eu des problèmes de compréhension
        surtout quant aux implémentations des autres parties. J'ai eu la chance
        d'avoir un groupe prêt à donner de son temps pour m'expliquer leur
        travail ce qui m'a permis de me sentir soutenu tout au long du projet.
        Pour ce qui est de ma propre partie, je n'ai pas eu de problèmes
        majeurs, si ce n'est le stress de bien l'avoir finie à temps.
        Heureusement, notre groupe s'est motivé pour travailler tous ensemble
        après les cours, ce qui m'a permis d'avancer et de vraiment ressentir
        une bonne ambiance de groupe, surtout lors de la dégustation du tacos
        qui suivit.

    \item \textbf{Adrian Rivoire-Galkiewicz} :

        En tant que débutant en C, notre chef de projet m’a assigné une tâche
        assez facile. J’ai passé beaucoup de temps à comprendre et apprendre à
        utiliser les différentes bibliothèques pour le pré-traitement et
        l’interface, et grâce à cela j’ai pu réaliser ce qui m’était demandé en
        temps voulu. Nous avons une très bonne entente de groupe et ainsi, nous
        avançons à une vitesse raisonnable dans la réalisation de ce projet.
        Aussi, le fait que nous ne soyons pas tous des débutants dans le groupe
        est un point très positif. En effet, pour le projet de S2, mon groupe et
        moi nous étions retrouvés dans une situation où nous ne savions pas quoi
        faire pour démarrer le projet et nous avons ainsi perdu beaucoup de
        temps à rechercher différentes informations sur Internet. Pour ce
        projet, Thibault a été en capacité de nous assigner des tâches précises
        et cela m’a permis de m’organiser dans mon travail. Je suis certain que
        nous allons remplir les objectifs de ce projet et que celui-ci nous
        permettra d’apprendre beaucoup de choses, notamment sur le C.

\end{itemize}

\chapter{Conclusion}

Notre première partie du projet se termine donc sur un ressenti général très
positif vis-à-vis de notre OCR. La structure du réseau de neurones est en place,
et ce dernier est suffisament modulaire pour accueillir l'apprentissage requis
pour la reconnaissance de caractères. La suppression des couleurs avec la
binarisation est fonctionnelle, ce qui constitue une première phase majeure de
pré-traitement de l'image. La segmentation d'une image pré-formatée est déjà
bien avancée et détecte les lignes et caractères. Enfin, un début d'interface a
été posé pour nous faciliter le travail lors de nos tests avec des composantes
visuelles.

Il reste cependant de nombreux points à résoudre et à implémenter d'ici la
prochaine soutenance comme l'approche par \textit{mini-batch} pour l'algorithme
du gradient, ou encore toute la partie pré-traitement de l'image pour la
redresser et s'assurer que nos algorithmes de segmentation puissent correctement
détecter les lignes et caractères. L'apprentissage pour le réseau va aussi être
l'un des objectifs majeurs pour la soutenance finale, mais nous sommes confiants
face à cette difficulté.

D'un point de vue du groupe ce fut notre seconde vraie expérience de travail en
commun pour chacun d'entre nous, et malgré les obstacles nous avons réussi à
progresser et à nous plaire dans cet apprentissage.

\chapter{Références}

\paragraph{\bfseries{Réseau de neurones}}

\begin{itemize}
    \item Explication et détails d'implémentation \\
    \url{http://neuralnetworksanddeeplearning.com/}
    \item Intuition et fonctionnement \\
    \url{https://www.youtube.com/playlist?list=PLZHQObOWTQDNU6R1_67000Dx_ZCJB-3pi}
\end{itemize}

\paragraph{\bfseries{Pré-traitement \& Manipulation d'images}}

\begin{itemize}
    \item SDL \\ \url{https://www.libsdl.org/}
    \item Méthode d'Otsu \\ \url{http://www.ipol.im/pub/art/2016/158/}
\end{itemize}

\paragraph{\bfseries{Interface graphique}}

\begin{itemize}
    \item GTK+ \\ \url{https://www.gtk.org}
    \item Glade \\ \url{https://glade.gnome.org/}
\end{itemize}
