\chapter{Introduction}

Durant cette première étape de projet, nous avons beaucoup appris tant sur le
plan pratique que sur celui du travail en groupe. Tout d'abord, ce projet étant
bien plus orienté que celui de SUP, il est clair que nous partions de très loin
en termes de connaissances sur les réseaux de neurones. De plus, la majorité de
notre groupe (3 personnes sur 4) n'avait encore jamais codé en C.

Par conséquent, une longue phase de documentation à propos de la théorie de
l'OCR mais aussi de son implémentation en C, ainsi que du langage en lui-même a
été nécessaire. Ceci était primordial mais a donc légèrement retardé la phase
d'implémentation.

Malgré ce faible retard au début du projet nous avons réussi à avancer chacun
nos parties afin d'atteindre les objectifs de la première soutenance.

Nous avons également pris le temps de s'expliquer entre nous nos parties
respectives afin d'être sûr que chacun ait pu avoir un apprentissage pratique
comme théorique du projet dans sa globalité.

Nous exposerons dans ce rapport l'avancement du logiciel d'OCR en tant que
groupe ainsi qu'en tant qu'individus, puis expliquerons nos objectifs pour la
prochaine soutenance.

\begin{figure}
    \includegraphics[width=1.1\textwidth]{calvin}
    \caption*{\textit{Calvin and Hobbes}, Bill Watterson}
\end{figure}

\chapter{Présentation de l'équipe}

\begin{itemize}
    \item \textbf{Thibault Allançon (Chef de projet)} :

        Prolotiste jusqu’au-boutiste et peu pacifiste. Heureusement pour lui,
        Thibault est plus à l'aise en C qu'en rimes. D'ailleurs, il adore parler
        de lui à la 3ème personne, ce qui le rend entièrement apte à diriger ce
        projet de manière optimale (la preuve sera laissée comme exercice au
        lecteur).

    \item \textbf{Timothé Helme-Guizon} :

        Amateur d'animés, jeux vidéo, et programmation, ce jeune homme semblait
        paré à attaquer le projet d'OCR. Malheureusement, sa compréhension du
        réseau de neurones fut impossible, car on ne peut comprendre cette
        notion en ne possédant qu'un seul et unique neurone.

    \item \textbf{Quentin Le Helloco} :

        Autostoppeur intergalactique, il s'intéresse à l'intelligence
        artificielle afin de faire de Skynet une réalité. L'OCR semble être un
        bon moyen de commencer sa destruction de l'humanité (lorsqu'il n’est
        n'est pas trop occupé à jouer au football avec des hot-wheels).

    \item \textbf{Adrian Rivoire-Galkiewicz} :

        Maître magicien à ses heures perdues, il a déjà réussi à faire
        disparaître le nez d’un enfant. On le cherche toujours (l'enfant, pas le
        nez). Devenu pro dans l’art de la manipulation, il a réussi à faire
        croire pendant 18 années que son vrai nom était Adri\textbf{A}n alors
        que tout le monde sait qu’il n’existe que la forme avec un \textbf{E}.
        La légende raconte même qu'il est parvenu à infiltrer l'EPITA.
\end{itemize}

\chapter{Répartition des tâches}

\begin{center}
    \begin{tabular}{@{} l *4c @{}}
        \toprule
        \multicolumn{1}{c}{} &
            \textbf{Thibault} & \textbf{Timothé}  &
            \textbf{Quentin} & \textbf{Adrian} \\
        \midrule
        Réseau de neurones & R & & S & \\
        Suppression couleurs & & & & R \\
        Pré-traitement & S & S & R & \\
        Segmentation & S & R & & \\
        Interface & & & S & R \\
        \bottomrule
        \multicolumn{4}{l}{\footnotesize R = responsable, S = suppléant}\\
    \end{tabular}
\end{center}

\vspace{2em}

\begin{center}
    \begin{tabular}{@{} l *4c @{}}
        \toprule
        \multicolumn{1}{c}{} & \textbf{Soutenance 1}  & \textbf{Soutenance 2} \\
        \midrule
        \textbf{Réseau de neurones} \\
        Structure du réseau & 80\% & 100\% \\
        Apprentissage & 60\% & 100\% \\
        Jeu de données & 45\% & 100\% \\\\
        \textbf{Pré-traitement} \\
        Suppression des couleurs & 90\% & 100\% \\
        Formatage de l'image & 0\% & 100\% \\\\
        \textbf{Segmentation} & 55\% & 100\% \\\\
        \textbf{Interface} & 30\% & 100\% \\
        \bottomrule
    \end{tabular}
\end{center}
