\chapter{Expériences personnelles}

\section{Thibault Allançon}

Le langage C était mon premier langage de programmation, je l’apprécie donc tout
particulièrement d’autant plus que nous l’utilisons pour un projet assez
conséquent. En tant que chef de projet, j’ai commencé par conseiller mes
coéquipiers sur les différentes tâches que nous avions à effectuer. Durant la
première partie du projet, j’ai beaucoup travaillé sur le réseau de neurone,
mais j’ai également su me rendre disponible afin d’aider les autres sur leurs
parties. À la fin de la première soutenance, notre OCR fut bien avancé, et le
code derrière était vraiment de bonne qualité. Il y a une grosse différence
entre comprendre un réseau de neurones et en implémenter un de toutes pièces,
cette expérience fût donc très formatrice. J’ai pu aider mes camarades de groupe
grâce aux connaissances que je possédais déjà sur le langage C, ce qui nous
permit de dépasser certains obstacles et aller plus loin pour faire des bonus.
Au final, je suis fier de notre groupe qui a su produire un OCR tout à fait
fonctionnel !

\section{Quentin Le Helloco}

Au début de ce projet, je n’avais aucune connaissance en C, ni en réseau de
neurones. En revanche, je souhaitais m’orienter dans le secteur de
l’intelligence artificielle après mes études, c’est pour cela que j’ai voulu
m’occuper du réseau de notre projet. Les notions abordées dans ce thème sont
parfois très abstraites et très mathématiques, ce qui les rend difficiles à
assimiler. Avec Thibault comme chef de groupe, qui connaît déjà bien le C et le
réseau de neurones, j’ai pu apprendre beaucoup. Je sais maintenant davantage à
quoi ressemble cet aspect de l’informatique qui était assez flou auparavant.
J’ai grâce à cela accepté un stage dans le machine learning pour en apprendre
plus dans ce domaine. Ce projet me permettra d’avoir des bases lors de mon
stage.

Durant cette deuxième période de travail, j’ai trouvé plus de plaisir à avancer
ce projet car j’ai pu améliorer mes compétences en C, surtout en ce qui concerne
les pointeurs. Mon travail durant cette seconde période était d’optimiser et
d'entraîner le réseau de neurones. Notre chef de groupe connaissant très bien le
langage C, il a pu être d’une grande aide en cas de difficulté. Ce projet fut
très différent de celui de S2, notamment car nous n’avions pas à gérer les
contraintes d’un logiciel tel que Unity, et l’OCR se rapproche plus du type de
projet que l’on pourrait faire dans notre futur travail. Ce projet m’a aussi
permis d’appréhender plus en détail ce qu’était le machine learning et cette
vision plus concrète me permettra certainement de choisir plus facilement ma
majeure.

\section{Timothé Helme-Guizon}

N’ayant jamais fait de C auparavant, ce projet commençait déjà comme un
challenge pour moi. Heureusement, j’ai pu travailler sur une partie où
l’algorithme en lui-même était complexe, et non pas son implémentation en C.
J’ai donc pu avancer petit à petit sur l’implémentation de mon algorithme en le
testant au fur et à mesure. Cependant, celui-ci nécessitait beaucoup de lecture
de documentation, ce qui a pu parfois me décourager. En effet, étant assigné à
la segmentation, j’ai dû prendre le temps de découvrir les fonctions d’analyse
de l’image proposées par la librairie SDL. Ne maitrisant pas le C, de nombreux
segfault ou des problèmes quant à l’implémentation de certaines structures en
mémoire. Malgré tout, l’ambiance d’un groupe déterminé à avancer m’a permis de
me motiver, et de mener à bien ma partie. En plus de cette simple motivation,
j’ai pu demander de l’aide au reste de mon groupe en cas de problème, ce qui
aide vraiment à combattre le stress. S’ajoute à cela les quelques réunions de
groupes pour travailler ou parler du projet, où l’ambiance passait
harmonieusement du sérieux à  la blague.

\section{Adrian Rivoire-Galkiewicz}

Au commencement du projet, à l’instar de l’année dernière, je n’avais aucune
idée de comment nous allions réaliser celui-ci. Je me disais que l’histoire
allait se répéter. Fort heureusement, nous nous sommes très bien entendus, bien
que nous nous ne connaissions pas tous. Assez rapidement, nous nous sommes
concertés pour “évaluer” le niveau de chacun afin de pouvoir répartir les
tâches. Notre chef de projet, Thibault, nous a alors assigné des tâches
précises. De nature assez désorganisé, cela m’a permis de me fixer des objectifs
en un temps donné. Premièrement, il fallait se renseigner, comprendre comment et
pourquoi utiliser telle bibliothèque plutôt qu’une autre mais aussi découvrir le
C. Grâce aux TPs hebdomadaires et à Internet, j’ai facilement été en capacité de
comprendre les différentes méthodes que j’allais utiliser. Ensuite, il a fallu
mettre la main à la pâte. J’ai alors été en capacité de remplir le travail qu’on
m’avait assigné. De plus, au moindre souci majeur, Thibault était toujours
disponible pour m’aider à traiter celui-ci. J’ai beaucoup apprécié travailler
dans ce groupe. Nous nous entendons très bien et chacun a pu visualiser le
travail des autres afin de mieux comprendre comment notre OCR fonctionne dans
son intégralité. Je pense avoir beaucoup appris, principalement sur le C et les
quelques bibliothèques utilisées pour ce projet, mais aussi sur le travail
d’équipe ainsi que l’importance de l’organisation dans un projet de groupe. En
dehors du côté scolaire, j’ai beaucoup apprécié travailler sur ce projet, qui
fût une expérience très enrichissante.

\chapter{Conclusion}

Nous sommes tous très heureux du résultat de ce projet car, malgré le manque de
connaissance au démarrage de celui-ci, nous avons réussi à rendre un OCR
fonctionnel. Ce dernier est constitué d'un pré-traitement complet de l'image
avec passage en noir et blanc, puis binarisation, redressement et réduction du
bruit. La segmentation prend ensuite le relai pour fournir les caractères
individuels au réseau de neurones qui se charge de la restitution du texte.

Cet OCR est évidemment perfectible car avec un apprentissage plus poussé dans
plusieurs polices, il pourrait davantage généraliser la reconnaissance. Un
affinage des fonctions de pré-traitement comme la réduction du bruit permettrait
d'améliorer le rendu et la qualité de l'OCR. Nous restons cependant tous très
satisfaits du résultat final qui remplit les différents objectifs que nous nous
étions imposés.

Notre groupe a été et reste encore très soudé. Chacun a su être disponible pour
les autres, tant pour de simples explications que pour des conseils divers et
variés. En conclusion, nous pouvons affirmer que nous sommes fiers de ce projet
et que nous avons tous progressé, tant dans notre organisation du travail de
groupe que dans nos connaissances en programmation C.

\chapter{Références}

\paragraph{\bfseries{Réseau de neurones}}

\begin{itemize}
    \item Explication et détails d'implémentation \\
    \url{http://neuralnetworksanddeeplearning.com/}
    \item Intuition et fonctionnement \\
    \url{https://www.youtube.com/playlist?list=PLZHQObOWTQDNU6R1_67000Dx_ZCJB-3pi}
    \item Améliorations du réseau de neurones \\
    \url{http://neuralnetworksanddeeplearning.com/chap3.html}
\end{itemize}

\paragraph{\bfseries{Pré-traitement}}

\begin{itemize}
    \item Méthode d'Otsu \\ \url{http://www.ipol.im/pub/art/2016/158/}
    \item Redressement de l'image
        \begin{itemize}
            \item \url{http://leptonica.org/skew-measurement.html}
            \item \url{https://people.eecs.berkeley.edu/~fateman/kathey/skew.html}
        \end{itemize}
    \item Réduction du bruit \\
    \url{https://en.wikipedia.org/wiki/Noise_reduction#Removal}
\end{itemize}

\newpage

\paragraph{\bfseries{Interface graphique \& Manipulation d'images}}

\begin{itemize}
    \item SDL
        \begin{itemize}
            \item \url{https://www.libsdl.org/}
            \item \url{https://www.libsdl.org/projects/SDL_image/}
        \end{itemize}
    \item GTK+
        \begin{itemize}
            \item \url{https://www.gtk.org}
            \item
                \url{https://developer.gnome.org/gtk3/stable/gtk-getting-started.html}
        \end{itemize}
    \item Glade \\ \url{https://glade.gnome.org/}
\end{itemize}
